\chapter{Metodologia}

    Considerando o objetivo de combinar os dados do acelerômetro, giroscópio e magnetômetro para obter um dado mais preciso da movimentação causada pelo ciclo cardíaco, o primeiro passo será realizar essa combinação em um ambiente de testes a fim de avaliar os sensores e os algoritmos a serem utilizados e definir se esse método é praticável.

    Antes de considerar as possibilidades de sensoriamento é importante considerar os algoritmos e os dados que esses necessitam. As técnicas que geralmente realizam esse processo utilizam de algoritmos iterativos, que combinam dados de acelerômetros, giroscópios e magnetômetros para atualizar uma estimativa da orientação, considerando um corpo rígido. Esse algoritmo iterativo é geralmente um filtro de Kalman estendido (EKF) baseado em "quaternions" %Colocar alguns artigos
 , mas existem outras alternativas baseadas em diferentes estimadores que também são estudados para esse efeito %Colocar mais artigos
, a fim de conseguir estudar uma variedade de técnicas para conseguir a melhor precisão no caso estudado é necessário um sensor, ou um conjunto de sensores, capaz de realizar a medição de acelerômetro, giroscópio e magnetômetro, com frequência constante e com baixo efeito inercial, dado que as vibrações do sistema cardíaco são pequenas.%Adicionar uma ordem de grandeza
    
    Considerando essas restrições a escolha inicial do sensor será o MPU-9250 da InvenSense (https://www.invensense.com/products/motion-tracking/9-axis/mpu-9250/), esse contêm um acelerômetro, um giroscópio e um magnetômetro, todos de três eixos, com precisão de $16384 LSB/g$, $131 LSB/(º/s)$ e $1,67 LSB/uT$ e ruído de $400 ug/sqrt{Hz}$ e $0.033 º/s-rms$\cite{MPU9250}, e custo de aproximadamente R\$18. Esse sensor foi escolhido baseado na sua grande disponibilidade, na suas características de ruído e precisão, que estão na mesma linha dos acelerômetros utilizados em grande parte dos trabalhos da área, e na sua capacidade de realizar leituras para memória interna, permitindo assim leituras com frequência bem definida. %Talvez citar alguns aqui

    Para estudar o sensor será utilizado um Arduino que será responsável por passar os dados para um computador, que armazenará esses dados. A análise desses dados será feita depois da obtenção, de maneira offline, para que os algoritmos possam ser estudados e corrigidos independentemente do experimento. Os algoritmos estudados serão inicialmente focados no uso do filtro de Kalman estendido, mas, caso necessário, serão expandidos para outras técnicas, a fim de localizar o melhor método de combinação dos sensores, levando em consideração as dimensões reduzidas desses, e a provável existência de movimentos de maior magnitude, como o paciente andando, que vão influenciar fortemente nos resultados.

    Nessa etapa serão realizados experimentos considerando movimentos da mesma ordem de grandeza dos movimentos cardíacos, de forma a se obter dados numéricos da precisão desses métodos quanto a detecção ao longo do tempo. O objetivo final dessa etapa é determinar se o sensor escolhido é capaz de realizar as medições necessárias e determinar qual é a precisão que os algoritmos conseguem atingir, tanto para curtas durações, período de até dois batimentos cardíacos, como para longas durações.  

    Uma vez determinado o sensor e o algoritmo de combinação dos dados, a etapa de análise dos resultados pode ser iniciada, nesse momento o sensor de 9 eixos será colocado na região cardíaca, na posição central%USAR NOMES MELHOR PARA ESSA POSIÇÃO!!!!!!!!
, em conjunto com sensores de referencia, ECG ou fonocardiograma, e os dados de todos eles serão obtidos de forma sincronizada. Considerando que nesse momento cada amostra será formada por aproximadamente por 22 bytes de dados será necessário utilizar um microcontrolador mais capaz.

    Considerando o desejo de que esse sensor seja capaz de funcionar continuamente, o microcontrolador KL25Z-AGMP01 da NXP (http://www.nxp.com/products/sensors/gyroscopes/10-axis-sensor-data-logger-reference-design:RD-KL25-AGMP01) é um bom candidato, uma vez que ele foi criado com o objetivo de ser um coletor de dados, ele já contém bateria recarregável e entrada para microSD, permitindo a aquisição de dados por longos períodos. Além dessas qualidades ele também contém um sensor de 9 eixos embutido, que pode ser utilizado para remoção de movimentos do paciente, considerando que ele esteja na região torácica, mas distante do coração.

    Com esse novo conjunto de dados podemos realizar uma análise sobre os dados do sensor de 9 eixos, comparando ele com os sensores de referência, de forma a definir as informações obtidas por esses. Inicialmente será observado se a análise do sensor de 9 eixos é capaz de obter o batimento cardíaco com precisão e robustez suficiente, depois será estudado se ele é capaz de diferenciar as etapas e eventos do ciclo cardíaco, como o abrir e fechar das válvulas, ou os momentos de sistole e diástole do ventrículo e átrio. Sobre esses resultados serão estudados que outras informações podem ser obtidos, como estimação da variação de pressão entre cada ciclo cardíaco, do volume de sangue ejetado, entre outros.

    Além disso também será estudado a capacidade de se obter essas informações de modo contínuo, ou seja, com o paciente em movimento normal na sua rotina. Para essa etapa será utilizado o sensor de 9 eixos fora da região cardíaca para se estimar o movimento geral do corpo e remover esse dos dados.

    Por final espera-se obter uma definição da factibilidade e utilidade desse método de sensoriamento da condição cardíaca por meio de localização da posição torácica, além de obter um estudo detalhado sobre o uso do sensor de 9 eixos para a localização em pequena escala. 
