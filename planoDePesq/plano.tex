- Utilizar dados de acelerômetro em conjunto com dados de giroscópio para realizar análise dos dados do seismeocardiogram
    - Problema com localização do batimento utilizando somente o 
    - Mostrar que muitos acelerômetros comerciais contêm um giroscópio e portanto esse dados não deve ser ignorado
    - Procurar outras referências de acelerômetro + giroscópio 
    - Achar uma boa referência do filtro de Kalman com quatérnions
        - Explicar rapidamente
        - Achar métodos alternativos
    - Citar o paper sobre respiração



- é importante usar isso, pois a superfície do torax é muito irregular, portanto  a direção do eixo Z, geralmente usado, muda de pessoa para pessoa ou entre diferentes posições, isso pode ser causador da grande variabilidade inter paciente , mesmo durante um mesmo experimento, como o movimento é de vibração se torna muito importante acompanhar o ângulo do acelerômetro para conseguir acompanhar a real diração que cada um dos seus eixos está apontando
    
    
    
         
