\chapter{Objetivo}

    O objetivo desse trabalho é pesquisar técnicas de sensoriamento cardíaco utilizando uma combinação de dados de diferentes sensores para obter informações sobre o ciclo cardíaco de maneira contínua e confiável, tentando inicialmente replicar a confiabilidade do ECG e depois expandir, tentando obter dados mais relevantes sobre o estado do ciclo cardíaco.

    Como discutido na sessão anterior o sismeocardiograma está sendo fortemente estudado, devido à melhoria das tecnologias de acelerômetros e microcontroladores, mas geralmente os dados desse sensor são estudados independente do seu significado físico, sendo observados somente como um sinal elétrico. 
    
    Grande parte das pesquisas leva em consideração somente os dados de um dos eixos de um acelerômetro triaxial, considerando que esse representa a direção dorso-ventral, que deveria onde os movimentos mais relevantes se encontram, mas esse uso simplificado falha em desconsiderar as possíveis variações na direção do acelerômetro. 
    
   Essas variações podem ser resultantes da posição inicial do sensor, que devido a superfície não uniforme da região torácica poderia não estar na direção correta, ou causadas pela variação da direção decorrente do movimento, que, por ser de natureza vibratória, tende a causar alterações na direção do acelerômetro, torando necessário uma correção dessa. %Talvez citar o trabalho de respiração aqui e preciso confirmar esse rolê da vibração

    Mesmo considerando que a posição inicial seja definida com cuidado e que as mudanças de direção não sejam suficientes para gerar uma variação nos dados, a utilização de somente um dos eixos de movimento, ou a utilização dos três independentemente, como é feita em alguns trabalhos, falha em observar o dado pelo que ele é originalmente, e portanto tende a perder informações significativas que podem aparecer ao se levar em consideração que esse dado vem de uma movimentação física de uma sensor com massa e dimensões conhecidas. 

    Portanto proponho utilizar um sensor completo de posicionamento para realizar um seismeocardiograma, analisando a variação de posição do sensor colocado sobre a região torácica ao longo do tempo. Esse sensor seria composto de um acelerômetro, um giroscópio e um magnetômetro, todos triaxiais, e estaria ligado a um microcontrolador, que seria responsável por continuamente ler e armazenar esses dados. 

    Os dados desse sensor serão combinados com o objetivo de gerar uma visão mais completa do movimento cardíaco, tentando reduzir as incertezas e variações apresentadas pelo seismeocardiograma padrão. Essa combinação será feita por algoritmos de filtragem adaptativa como filtros de Kalman e outros, que levam em consideração as origens físicas de cada sensor para obter a posição, mas também por análises matemática, que comparam os dados com um objetivo, por exemplo batimentos cardíacos, e realizam análises iterativa para se localizarem funções que combinam os dados para atingir o objetivo.

    Serão estudados diversas posições para esse sistema, tentando obter informações sobre o movimento cardíaco, colocando sobre a região torácica, informações sobre pressão arterial, colocando sobre artérias próximas a superfície, e algumas outras com o objetivo de obter o maior número de informações complementares sobre o ciclo cardíaco.
